\documentclass[12pt]{article}
\usepackage[T1]{fontenc}
\usepackage[utf8]{inputenc}
\usepackage{txfonts}
\usepackage[paperheight=29.7cm,paperwidth=21.0cm,left=2.54cm,right=2.54cm,top=2.54cm,bottom=2.54cm]{geometry}

\setlength\parindent{0pt}
\renewcommand{\arraystretch}{1.3}
\begin{document}
\begin{center}
{\LARGE Moravec's paradox}
\end{center}


\begin{center}
{\large Pushkin Rathore 18111043}
\end{center}


\begin{center}
{\large July, 2021}
\end{center}


\vspace{1\baselineskip}
It seems almost paradoxical to suggest that a technology ruled by logic — such as AI — could fall prey to paradoxes. And yet, artificial intelligence isn’t immune to the occasional illogical truth. Moravec’s paradox is one such contradictory reality within the development of AI. But as artificial intelligence grows in ability, how has this paradox influenced its development? Here, we take a closer look at Moravec’s paradox and what it means for modern AI.
\\

Moravec’s paradox is a phenomenon surrounding the abilities of AI$-$powered tools. It observes that tasks humans find complex are easy to teach AI. Compared, that is, to simple, sensorimotor skills that come instinctively to humans.
\\

\textit{$``$It is comparatively easy to make computers exhibit adult level performance and difficult or impossible to give them the skills of a one$-$year$-$old.$"$}
\\

For example, artificial intelligence can complete tricky logical problems and advanced mathematics. But the ‘simple’ skills and abilities we learn as babies and toddlers — perception, speech, movement, etc. — require far more computation for an AI to replicate. In other words, for AI the complex is easy, and the easy is complex. The history of AI has seen an impact from Moravec’s paradox. In fact, it’s arguably a factor that held back development and contributed to the AI effect.
\\

The AI effect is a phenomenon that has seen AI$-$powered tools lose their ‘AI’ label over time, due to not being ‘true’ intelligence. Moravec’s paradox could have contributed to this. That is, the reason these tools lost their ‘intelligent’ status is that the tasks it does are simple, once you break them down.  No matter how good AI tools and programs got at games and logic, thanks to Moravec’s paradox, they couldn’t complete ‘basic’ human tasks. How could anything that can’t replicate the behaviour of a toddler be ‘intelligent?’
\\

Moravec’s paradox explains why AI capable of adult$-$level reasoning is old hat, but AI vision, listening, and learning is new and exciting. Indeed, things are changing. For instance, we are beginning to see AI tools like image classification and facial recognition — that is, a machine’s equivalent of sight.  Meanwhile, personal assistants like Alexa are an example of AI becoming capable of ‘hearing’ and understanding us. This is thanks to natural language processing (NLP). Similarly, AI is becoming capable of speech, as with these assistants, or advancements like Google Duplex.\end{document}
